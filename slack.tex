\subsection{Slack}

The data management group uses instant messaging tools to answer questions, 
share code snippets and errors, and coordinate activities in real-time.  
Before using instant messaging, we would communicate primarily via e-mail or 
over the phone. While these tools do still have their place, being able to have 
conversations in a team-wide application has been very helpful. Discussions
happen with more team members participating, instead of just who happened to
be on a phone call or an e-mail thread. Questions and concerns have been able
to be answered much more quickly.  Site channels help coordinate discussions
useful to local teams.  We even have non-work related channels were we can
discuss various topics, which has sparked conversations about previously
unrealized common interests. It has been beneficial to team building and 
helped foster friendships.

We have used two different instant messaging tools in the last four years. The
first tool we used was HipChat. Besides the essential features expected
from an instant messaging client (such as discussion channels and one
to one direct messaging), we were also able to receive Jenkins and JIRA issue
notifications, which we found very helpful.

We used HipChat for two years and eventually started looking at other options 
because we found the user experience was, at times, inconsistent and 
frustrating. One of the most significant issues was how it managed unread
messages. If the user left the application, HipChat did not maintain where the
user was in the message chain. When the tool restarted, it required the user
to scroll back through messages to pick up where they left off. Notifications
were erratic, and sometimes would arrive late or not at all. Some users
particularly liked features such as video chat, while other users found this
feature unstable and resorted to using other video chat services such as Google 
Hangouts. 

After a brief testing period, we decided to transition to Slack, which we
continue to use today. The move from HipChat to Slack was straightforward,
although we did lose the discussion history from HipChat.  At first, we 
thought this would be an issue, but in practice, we rarely search history older
than a couple of weeks.  We have had difficulties in balancing when to 
archive these discussions for the community.  We've also had some problems
making sure that decisions made within a Slack conversation were
distributed to more team members, since everyone may not have been in the 
channel at that time.  We continue to improve our efforts in this regard by 
starting RFC's or Confluence discussions to distribute information more widely.

Slack initially didn't have video chat when we switched from HipChat, but
it does now.  We use this for ad-hoc calls for the channel, which has been
easier than coordinating a call via Google Hangouts or BlueJeans.

When we made the switch, we wanted to take the opportunity to normalize the
users' chat names to match the names used on GitHub, which would have 
simplified how we notified users of software build failures.  It proved
difficult to coordinate, so we let our Jenkins bot do the translation between
their GitHub ID and their Slack account.

Slack offers of HipChat's functionality and improves several features. Slack 
has several options on how unread messages are handled and displayed, allowing
users to choose how the application behaves when the application restarts. 
Users can decide how to handle unread messages either showing them where the
application was last used, or by marking them already read.

Other features include direct messaging can be set up for small groups,
instead of creating a public chat room for an ephemeral discussion topic.
We create temporary channels for members to communicate during workshops and 
conferences, which helps team members coordinate while on-site and to involve
people who aren't attending.  This paper itself was even organized
within a Slack channel, allowing us to discuss it and to get automatic
notifications of changes to it via the GitHub Slack bot.

Previews of URLs, global and per channel notification preferences and an 
API to integrate third-party or custom chat-bot apps. One of those apps, 
sqrbot, is discussed in the next section.  

Other subsystems within LSST which either had already been using HipChat or
were beginning to explore instant messaging services have also transitioned to 
Slack. We expect this to be of great benefit to us as we integrate the work 
from the LSST sub-systems.

\subsubsection{Chat Ops}

DM-SQuaRE is using a Slack chatbot called ``sqrbot'' to make some tasks
easier. Currently it performs a range of functions, from returning the
status of various infrastructure machines to creating technotes to
monitoring whether metrics in our processing stack have changed between
CI builds; it also listens for mentions of Jira tickets, and when it
hears one, posts a link to that ticket.

All of these tasks are frequently requested actions that formerly
required breaking workflow.  Either (as with technote creation) the
requestor would have to interrupt someone else in order to get the work
performed, or they'd have to interrupt their own workflow to, for
instance, copy the ticket name, go to Jira, and search for the ticket.
With sqrbot, the requestor can simply ask sqrbot in a Slack channel for
the information (or creation of a technote skeleton, etc.) and get
immediate gratification without needing to task-switch or interrupt
a co-worker.

The basic architecture is simple: sqrbot is simply a collection of hubot
scripts running as a Slack bot, which in turn drive microservices,
written in Python and implemented using the Flask framework.  These
microservices have an API that responds with JSON, so the job of sqrbot
is simply to accept commands, create appropriate HTTP transactions, and
then reformat the output into a conversational format.  The whole
assembly runs in a Kubernetes cluster.

