\section{INTRODUCTION}

The data management system\cite{2015arXiv151207914J} for the Large Synoptic Survey Telescope\cite{2008arXiv0805.2366I} has been under development since at least 2004\cite{2004AAS...20510811A}.
During that time a number of technologies have been adopted and our development practices have evolved as we transitioned from the research and development phase to construction.
The Data Management (DM) team is distributed, with representation from Princeton University, the University of Washington, the National Center for Supercomputing Applications at Urbana-Champaign, IPAC in Pasadena, the LSST Project Office in Tucson, and the SLAC National Accelerator Laboratory near Stanford University; along with some external contributions from CC-IN2P3 in Lyon, France.
Given this, it is important that comunication channels are open and easy to use and that our tools evolve as community standards evolve.
Being agile enough to be able to migrate from one tool to another during the lifetime of a project is key when the software, processes, and people, change over what will be a 25 year period once the 10-year survey completes.
For example, over the years we have migrated the codebase from Subversion to git; we have switched instant messaging from HipChat to Slack; we have migrated continuous integration from builbot on our own hosts to Jenkins running in the cloud; and we have moved documentation standards from Doxygen to Sphinx.

In the following sections we describe the current development practices for LSST DM.
